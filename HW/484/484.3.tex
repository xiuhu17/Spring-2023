\documentclass{article}
\usepackage[utf8]{inputenc}
\documentclass{article}
\usepackage{indentfirst}
\usepackage{geometry}
\usepackage{ntheorem}
\usepackage{amsmath}
\usepackage{amssymb}
\usepackage{mathtools}
\geometry{left=3cm,right=2.5cm,top=2.5cm,bottom=2.5cm}
% \indent
\newtheorem*{proposition}{Proposition}
\newtheorem*{definition}{Definition}
\newtheorem*{corrolary}{Corrolary}
\newtheorem*{consider}{Consider}
\newtheorem*{theorem}{Theorem}
\newtheorem*{suppose}{Suppose}
\newtheorem*{notice}{Notice}
\newtheorem*{define}{Define}
\newtheorem*{denote}{Denote}
\newtheorem*{lemma}{Lemma}
\newtheorem*{claim}{Claim}
\newtheorem*{proof}{Proof}
\newtheorem*{case}{Case}
\newtheorem*{skill}{Skill}
\newtheorem*{axiom}{Axiom}
\newtheorem*{algorithm}{Algorithm}
\def\theo{\begin{theorem}}
\def\ax{\begin{axiom}}
\def\alg{\begin{algorithm}}
\def\pro{\begin{proof} }
\def\cla{\begin{claim}}
\def\sk{\begin{skill}}
\def\prop{\begin{proposition}}
\def\defi{\begin{definition}}
\def\lem{\begin{lemma}}
\def\cor{\begin{corrolary}}
\def\den{\begin{denote}}
\def\define{\begin{define}}
\def\supp{\begin{suppose}}
\def\enu{\begin{enumerate} \end{enumerate}}
\def\RR{\mathbb{R}}
\def\ZZ{\mathbb{Z}}
\def\NN{\mathbb{N}}
\def\QQ{\mathbb{Q}}
\def\CC{\mathbb{C}}
\def\FF{\mathbb{F}}
\def\KK{\mathbb{K}}
\def\calBB{\mathcal{B}}
\def\calCC{\mathcal{C}}
\def\calLL{\mathcal{L}}
\def\calPP{\mathcal{P}}
\def\implies{\Longrightarrow}
\def\bfit#1{\textit{\textbf{#1}}}
\DeclareSymbolFont{largesymbolsA}{U}{txexa}{m}{n}
\DeclareMathSymbol{\varprod}{\mathop}{largesymbolsA}{16}
\title{Math 484}
\author{Zhihao Wang}
\date{Feb 2023}

\begin{document}

\maketitle

\subsection*{1 Proof}
\subsubsection*{(a)}

By the definition of convex set,
$$\forall (x_1, y_1), (x_2, y_2)  \in D, [(x_1, y_1), (x_2, y_2)] := \{t(x_1, y_1)+ (1 - t)(x_2, y_2) : 0 \le t \le 1 \}$$

Then we can get,
$$\forall (z_x, z_y) \in [(x_1, y_1), (x_2, y_2)], (z_x, z_y) := (tx_1 + (1 - t)x_2, ty_1 + (1 - t)y_2)$$

So, we have the equality,

$$|z_x| + |z_y| = |tx_1 + (1 - t)x_2| + | ty_1 + (1 - t)y_2)| $$

By triangular inequality, 
\[
\begin{split}
|tx_1 + (1 - t)x_2| + | ty_1 + (1 - t)y_2)| &\le t(|x_1| + |y_1|) + (1-t)(|x_2| + |y_2|) \\
&\le t + (1 - t) \\
& = 1
\end{split}
\]

Therefore, $(z_x, z_y)  \in D$
\subsubsection*{(b)}
By the definition of convex set,
$$\forall x, y \in L_c^{-}(f), [x, y] := \{tx + (1 - t)y : 0 \le t \le 1 \}$$

Then we can get,
$$\forall z \in [x, y], z := tx + (1 - t)y$$

So, we have the equality,

$$f(z) = f(tx + (1-t)y)$$

Since the function $f$ is convex function, we have the inequality,
$$ f(tx + (1-t)y) \le tf(x) + (1-t)f(y)$$

Further,
\[
\begin{split}
 tf(x) + (1-t)f(y) &\le tc + (1-t)c \\
&= c
\end{split}
\]

So, we get $z \in  L_c^{-}(f)$ 
\subsection*{2 }
\subsubsection*{(a)}

I have,

$$g(x) = -x, f(x) = x^2$$

These two are convex since,

$$\forall x \in \mathcal{R}, Hg(x)= \begin{bmatrix} 0\end{bmatrix} \succeq 0$$

$$\forall x \in \mathcal{R}, Hf(x)= \begin{bmatrix} 2\end{bmatrix} \succ 0$$

And,

$$h(x) = g(f(x)) = -x^2$$

is not convex, since

$$\forall x \in \mathcal{R}, Hh(x)= \begin{bmatrix} -2\end{bmatrix} \prec 0$$
\subsubsection*{(b)}

I have,

$$g(x) = -e^{-x}, f(x) = x$$

$f(x)$ is convex since, 

$$\forall x \in \mathcal{R}, Hf(x)= \begin{bmatrix} 0\end{bmatrix} \succ 0$$

$g(x)$ is increasing since, 
$$\forall x \in \mathcal{R}, g'(x) = e^{-x} \ge 0$$

But, 

$$h(x) = g(f(x)) = -e^{-x} $$

is not convex, since

$$\forall x \in \mathcal{R}, Hh(x)= \begin{bmatrix} -e^{-x}\end{bmatrix} \prec 0$$

\subsubsection*{(c)}

I have,

$$g(x) = x, f(x) = -x$$

These two are convex since,

$$\forall x \in \mathcal{R}, Hg(x)= \begin{bmatrix} 0\end{bmatrix} \succeq 0$$

$$\forall x \in \mathcal{R}, Hf(x)= \begin{bmatrix} 0\end{bmatrix} \succeq 0$$

And,

$$h(x) = g(x)f(x) = -x^2$$

is not convex, since

$$\forall x \in \mathcal{R}, Hh(x)= \begin{bmatrix} -2\end{bmatrix} \prec 0$$

\subsection*{3 Proof}
\subsubsection*{(a)}
As we can see, $\mathcal{R}$ is a convex set. And we can use theorem 2.3 to prove this,

$$\forall x \in \mathcal{R}, Hf(x) = e^{e^{e^x}+e^x+x}\left(e^{e^x+x}+e^x+1\right)$$

Then we need to prove that $Hf(x)$ is positive semidefitive.
$$\forall u \in \mathcal{R}, u^THf(x)u = e^{e^{e^x}+e^x+x}\left(e^{e^x+x}+e^x+1\right) u^2 \ge 0$$

Therefore this is convex

\subsubsection*{(b)}
First split this function into two parts,
$$f(x, y, z) = p(x, y) + q(y, z) = (x + y)^4 + (y-z)^4$$

As we can see, $\mathcal{R}^3$ is a convex set. And we can use theorem 2.3 to prove this,

$$\forall x \in \mathcal{R}, Hp(x) = (x+2y)^2\begin{bmatrix} 1&1 \\ 1 & 4 \end{bmatrix}$$

By using the Sylvester's criterion, we find that $\Delta_1 = 1 > 0, \Delta_2 = 3 > 0$. Therefore, this is a positive definite matrix, therefore, $p(x)$ is convex.

$$\forall x \in \mathcal{R}, Hq(x) = 12(y-z)^2\begin{bmatrix} 1&-1 \\ -1 & 1 \end{bmatrix}$$

By using the eigenvalue test, we find that $\lambda_1 = 0 \ge 0, \lambda_2 = 2 > 0$. Therefore, this is a positive semidefinite matrix, therefore, $p(x)$ is convex.

Therefore, the original function $f(x, y , z)$ is convex.

\subsubsection*{(c)}
As we can see, $\mathcal{R}$ is a convex set. And we can use definition to prove this,

$$\forall x , y \in \mathcal{R}$$

We have the the inequality,

\[
\begin{split}
G(t) = tf(x) + (1 - t)f(y) - f(tx + (1-t)y), t \in [0, 1] 
\end{split}
\]

If $x \ge 0, y \ge 0$, then we know $tx + (1-t)y \ge 0$, 

$$G(t) = tx + (1 - t)y - tx - (1 - t)y \ge 0$$

If $x < 0, x < 0$, then we know $tx + (1 - t)y < 0$, 
$$G(t) = 0 - 0 \ge 0$$

WLOG, $x \ge 0, y < 0, (1-t)y \le 0$, and $f(x)$ is an increasing function,
$$G(t) = tx + 0 - f(tx + (1-t)y)\ge tx - f(tx) = tx - tx \ge 0$$

Therefore, this function is a convex function.

\subsection*{4 Proof}
By using AM-GM inequality, we know that,

$$xy^2 + yz^2 + zx^2 \ge 3 \sqrt[3]{xy^2yz^2zx^2} = 3 \sqrt[3]{(xyz)^3} = 3$$

which is subject to,

$$x > 0, y > 0, z > 0$$
$$xyz = 1$$

When, $$x = y = z = 1$$


\end{document}
