\documentclass{article}
\usepackage[utf8]{inputenc}
\documentclass{article}
\usepackage{indentfirst}
\usepackage{geometry}
\usepackage{ntheorem}
\usepackage{amsmath}
\usepackage{amssymb}
\usepackage{amsmath}% http://ctan.org/pkg/amsmath
\newcommand{\notimplies}{%
  \mathrel{{\ooalign{\hidewidth$\not\phantom{=}$\hidewidth\cr$\implies$}}}}
  
\geometry{left=3cm,right=2.5cm,top=2.5cm,bottom=2.5cm}
% \indent
\newtheorem*{proposition}{Proposition}
\newtheorem*{definition}{Definition}
\newtheorem*{corrolary}{Corrolary}
\newtheorem*{consider}{Consider}
\newtheorem*{theorem}{Theorem}
\newtheorem*{suppose}{Suppose}
\newtheorem*{notice}{Notice}
\newtheorem*{define}{Define}
\newtheorem*{denote}{Denote}
\newtheorem*{lemma}{Lemma}
\newtheorem*{claim}{Claim}
\newtheorem*{proof}{Proof}
\newtheorem*{case}{Case}
\newtheorem*{skill}{Skill}
\newtheorem*{axiom}{Axiom}
\newtheorem*{algorithm}{Algorithm}
\def\theo{\begin{theorem}}
\def\ax{\begin{axiom}}
\def\alg{\begin{algorithm}}
\def\pro{\begin{proof} }
\def\cla{\begin{claim}}
\def\sk{\begin{skill}}
\def\prop{\begin{proposition}}
\def\defi{\begin{definition}}
\def\lem{\begin{lemma}}
\def\cor{\begin{corrolary}}
\def\den{\begin{denote}}
\def\define{\begin{define}}
\def\supp{\begin{suppose}}
\def\enu{\begin{enumerate} \end{enumerate}}
\def\RR{\mathbb{R}}
\def\ZZ{\mathbb{Z}}
\def\NN{\mathbb{N}}
\def\QQ{\mathbb{Q}}
\def\CC{\mathbb{C}}
\def\FF{\mathbb{F}}
\def\KK{\mathbb{K}}
\def\calBB{\mathcal{B}}
\def\calCC{\mathcal{C}}
\def\calLL{\mathcal{L}}
\def\calPP{\mathcal{P}}
\def\implies{\Longrightarrow}
\def\bfit#1{\textit{\textbf{#1}}}
\DeclareSymbolFont{largesymbolsA}{U}{txexa}{m}{n}
\DeclareMathSymbol{\varprod}{\mathop}{largesymbolsA}{16}
\title{484}
\author{Zhihao Wang}
\date{Feb 2023}

\begin{document}

\maketitle

\subsection*{1}

We first consider the equation,

$$\frac{1}{3}\sin(-A) +\frac{1}{3} \sin(-B) + \frac{1}{3}\sin(-C)$$

Since $f(x) = \sin(-x) = -\sin(x), x \in [0, \pi]$, 

$$f''(x) = \sin^2(x) \ge 0$$

Therefore, $f(x)$ is convex, by the Jensen's Inequality,

$$\frac{1}{3}\sin(-A) +\frac{1}{3} \sin(-B) + \frac{1}{3}\sin(-C) \ge \sin(-\frac{1}{3}(A + B + C)) = -\sin(\frac{\pi}{3}) = - \frac{\sqrt{3}}{2} $$

Then, we can get,
$$\frac{1}{3}\sin(-A) +\frac{1}{3} \sin(-B) + \frac{1}{3}\sin(-C) = -\left[\frac{1}{3}\sin(A) +\frac{1}{3} \sin(B) + \frac{1}{3}\sin(C) \right]$$


Therefore, 
$$\left[\frac{1}{3}\sin(A) +\frac{1}{3} \sin(B) + \frac{1}{3}\sin(C) \right] \le \frac{\sqrt{3}}{2}$$

Then, 

$$\sin(A) +\sin(B) + \sin(C) \le\frac{3\sqrt{3}}{2} $$


\subsection*{2}

\subsubsection*{(a)}

$$\underset{\delta \in \mathcal{R}^n}{\text{maximize}} \ \  v(\delta) = \left(\frac{1}{\delta_1}\right)^{\delta_1}\left(\frac{1}{\delta_2}\right)^{\delta_2}\left(\frac{4}{\delta_3}\right)^{\delta_3}$$

\[
\begin{split}
\text{subject to}\ \ \ & \delta_1 - \delta_3 = 0 \\ 
& -\delta_1 + 4\delta_2 - \delta_3 = 0 \\ 
& \delta_1 + \delta_2 + \delta_3 = 1 \\
& \delta_1, \delta_2, \delta_3 > 0
\end{split}
\]


\subsubsection*{(b)}


\[
\begin{split}
& \delta_1 - \delta_3 = 0 \\ 
& -\delta_1 + 4\delta_2 - \delta_3 = 0 \\ 
& \delta_1 + \delta_2 + \delta_3 = 1 \\
& \delta_1, \delta_2, \delta_3 > 0
\end{split}
\]


Solve the equation, then we can find,

$$\delta_1 = \frac{2}{5}, \delta_2 = \frac{1}{5}, \delta_3 = \frac{2}{5}$$

\subsubsection*{(c)}

Based on the solution we find in previous question,

$$\frac{t_1t_2^{-1}}{\frac{2}{5}}=\frac{t_2^4}{\frac{1}{5}}=\frac{t_1^{-1}t_2^{-1}}{\frac{2}{5}}$$

Then we can get,

$$(0, 0), (-1, \sqrt[5]{-\frac{1}{2}}), (1, \sqrt[5]{\frac{1}{2}})$$

Since, $t_1, t_2 > 0$, therefore $t_1 = 1, t_2=\sqrt[5]{\frac{1}{2}}$

\subsection*{3}

\subsubsection*{(1)}

$$\underset{\delta \in \mathcal{R}^n}{\text{maximize}} \ \  v(\delta) = \left(\frac{1}{\delta_1}\right)^{\delta_1}\left(\frac{1}{\delta_2}\right)^{\delta_2}\left(\frac{3}{\delta_3}\right)^{\delta_3}$$

\[
\begin{split}
\text{subject to}\ \ \ & 2\delta_1 - \delta_2 + \delta_3 = 0 \\ 
& \delta_1 + \delta_2 + \delta_3 = 1 \\
& \delta_1, \delta_2, \delta_3 > 0
\end{split}
\]

\subsubsection*{(2)}

Solve the equation, 

\[
\begin{split}
\text{subject to}\ \ \ & 2\delta_1 - \delta_2 + \delta_3 = 0 \\ 
& \delta_1 + \delta_2 + \delta_3 = 1 \\
& \delta_1, \delta_2, \delta_3 > 0
\end{split}
\]

Set $\delta_1 = s$, we can get,

$$\delta_1 = s, \delta_2 = \frac{1 + s}{2}, \delta_3 = \frac{1 - 3s}{2}, s\in (0, \frac{1}{3})$$

Then we have,

\[
\begin{split}
\underset{\delta \in \mathcal{R}^n}{\text{maximize}} \ \  v(\delta) &= \left(\frac{1}{\delta_1}\right)^{\delta_1}\left(\frac{1}{\delta_2}\right)^{\delta_2}\left(\frac{3}{\delta_3}\right)^{\delta_3} \\
& = f(s)\\ 
& = \left(\frac{1}{s}\right)^{s}\left(\frac{1}{\frac{1 + s}{2}}\right)^{\frac{1 + s}{2}}\left(\frac{3}{ \frac{1 - 3s}{2}}\right)^{ \frac{1 - 3s}{2}} 
\end{split}
\]


\subsubsection*{(3)}

Based on the given information, we can get,

$$\frac{t^2}{\frac{1}{15}} = \frac{t^{-1}}{\frac{8}{15}} = \frac{3t}{\frac{2}{5}}$$

Therefore, $t = \frac{1}{2}$

\subsubsection*{(5)}

Suppose we have a linear system function, which best feats,

$$Ax \cong b$$

Where $A$ is,

$$A =\begin{bmatrix} -1 & 1\\ 0 &1\\ 1&1\\ 2&1\\3&1 \end{bmatrix}$$

Where $x$ is,

$$x = \begin{bmatrix} m \\ n \end{bmatrix}$$


Where $b$ is,

$$b =\begin{bmatrix} 2 \\ 1 \\3\\2\\1 \end{bmatrix}$$

Therefore, use pseudoinverse,

$$x = A^{\dagger}b = (A^TA)^{-1}A^Tb = \begin{bmatrix} -0.1 \\ 1.9 \end{bmatrix}$$

Therefore, we get,

$$y = -0.1x + 1.9$$

\end{document}
