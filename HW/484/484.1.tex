\documentclass{article}
\usepackage[utf8]{inputenc}
\documentclass{article}
\usepackage{indentfirst}
\usepackage{geometry}
\usepackage{ntheorem}
\usepackage{amsmath}
\usepackage{amssymb}

\geometry{left=3cm,right=2.5cm,top=2.5cm,bottom=2.5cm}
% \indent
\newtheorem*{proposition}{Proposition}
\newtheorem*{definition}{Definition}
\newtheorem*{corrolary}{Corrolary}
\newtheorem*{consider}{Consider}
\newtheorem*{theorem}{Theorem}
\newtheorem*{suppose}{Suppose}
\newtheorem*{notice}{Notice}
\newtheorem*{define}{Define}
\newtheorem*{denote}{Denote}
\newtheorem*{lemma}{Lemma}
\newtheorem*{claim}{Claim}
\newtheorem*{proof}{Proof}
\newtheorem*{case}{Case}
\newtheorem*{skill}{Skill}
\newtheorem*{axiom}{Axiom}
\newtheorem*{algorithm}{Algorithm}
\def\theo{\begin{theorem}}
\def\ax{\begin{axiom}}
\def\alg{\begin{algorithm}}
\def\pro{\begin{proof} }
\def\cla{\begin{claim}}
\def\sk{\begin{skill}}
\def\prop{\begin{proposition}}
\def\defi{\begin{definition}}
\def\lem{\begin{lemma}}
\def\cor{\begin{corrolary}}
\def\den{\begin{denote}}
\def\define{\begin{define}}
\def\supp{\begin{suppose}}
\def\enu{\begin{enumerate} \end{enumerate}}
\def\RR{\mathbb{R}}
\def\ZZ{\mathbb{Z}}
\def\NN{\mathbb{N}}
\def\QQ{\mathbb{Q}}
\def\CC{\mathbb{C}}
\def\FF{\mathbb{F}}
\def\KK{\mathbb{K}}
\def\calBB{\mathcal{B}}
\def\calCC{\mathcal{C}}
\def\calLL{\mathcal{L}}
\def\calPP{\mathcal{P}}
\def\implies{\Longrightarrow}
\def\bfit#1{\textit{\textbf{#1}}}
\DeclareSymbolFont{largesymbolsA}{U}{txexa}{m}{n}
\DeclareMathSymbol{\varprod}{\mathop}{largesymbolsA}{16}
\title{Math 484}
\author{Zhihao Wang}
\date{January 2023}

\begin{document}

\maketitle

\subsection*{1}
$$\text{minimize} \limits_{x, y \in \mathcal{R}} -x - 2y$$
$$\text{subject to} \ x^2 + y^2 - 1 \le 0, -x^2 - y^2 + 1 \le 0$$

$$x^2 \ge 0, y^2 \ge 0 \implies x \in [-1, 1], y \in [-1, 1]$$

Then we can get 
\begin{equation*}
-f(x, y) = -x - 2y =
\begin{cases}
-x - 2 \sqrt{1 - x^2} \\
-x + 2 \sqrt{1 - x^2} 
\end{cases}
\end{equation*}

Since we need to minimize the $-f(x, y)$, so we only need to analyze $-x - 2 \sqrt{1 - x^2}$

$$x \in [-1, 1], g(x) = -x - 2 \sqrt{1 - x^2}$$

Then we can get the first and second derivative

$$g'(x) = 1-\frac{2x}{\sqrt{1-x^2}}, g''(x) = \frac{2}{\left(-x^2+1\right)\sqrt{-x^2+1}}$$ 

They are continuous on the interval $x \in (-1, 1)$

$$g'(x_0) = 0 \implies x_0 = \frac{1}{\sqrt{5}}\quad$$

Since $x_0 \in (-1, 1), x \in (-1, 1), g''(x) > 0$, so we get the global minimizer. 

We also need to check the end point which is $x = -1, x = 1$.

$$x = \frac{1}{\sqrt{5}}\quad \implies f(x, y) = \sqrt{5}$$
$$x = 1 \implies f(x, y) = 1$$
$$x = -1 \implies f(x, y) = -1$$

We can get the maximum value is $\sqrt{5}$

\subsection*{2}
\subsubsection*{a}
$$f(x) = x^3 + x + 1$$ 
So we can get its first derivative,

$$f'(x) = 3x^2 + 1 \ge 1 \implies \forall x \in \mathcal{R}, f'(x) \ne 0$$

We can not find global or local minimizer

\subsubsection*{b}
$$f(x) = e^x - 2x$$
So we can get its first and second derivative,

$$f'(x) = e^x - 2, f''(x) = e^x$$

And we can see that both function is continuous on $x \in \mathcal{R}$

$$f'(x_0) = 0 \implies x_0 = \lg2$$

Since we can see that

$$\forall x \in \mathcal{R}, f''(x) > 0 $$

We know $x_0 = \lg 2$ is both (strict) global and local minimizer for $x \in \mathcal{R}$

\subsubsection*{c}
$$f(x) = \frac{x}{x^2 + 1}$$
Since $\forall x \in \mathcal{R}, x^2 + 1 > 0$, we know $f(x)$ is continuous on $\mathcal{R}$, so we can get its first derivative,


$$f'(x) = \frac{-x^2+1}{\left(x^2+1\right)^2}$$
Since $\forall x \in \mathcal{R}, x^2 + 1 > 0$, we know $f'(x)$ is continuous on $\mathcal{R}$, so we can get its second derivative,

$$f''(x) = -\frac{2x\left(-x^2+3\right)}{\left(x^2+1\right)^3}$$



We can get two critical points from first derivative function.
$$f'(x) = 0 \implies x_0 = -1, x_1 = 1$$
$$f''(-1) = \frac{1}{2} > 0$$
$$f''(1) = -\frac{1}{2} < 0$$

We know $x_0 = -1$ is strict local minimizer.

When $x = -1$, we get $f(x) = -\frac{1}{2}$
$$\forall x' \in \mathcal{R}, (x' + 1)^2 \ge 0 \implies x'^2 + 1 \ge -2x'\implies-\frac{1}{2} \le \frac{x'}{ x'^2 + 1}$$
So, $x = -1$ is also a global minimizer



\subsubsection*{3}
$$A = \begin{bmatrix}
4 & -6 \\
3 & -5
\end{bmatrix} - \lambda I = 
\begin{bmatrix}
4 - \lambda& -6 \\
3 & -5 - \lambda
\end{bmatrix}
$$

We can get 
$$\det(A) = (4 - \lambda)(-5 - \lambda) - (-6)(3) = 0 \implies \lambda_0 = 1, \lambda_1 = -2$$

Since we know that different eignvalues, their corrosponding eignvectors are independent, so this matrix must be diagonalizable. 

$$A = SD^{-1}S, D = \begin{bmatrix}
1 & 0 \\
0 & -2
\end{bmatrix}$$

We can get 
$$A = \begin{bmatrix}2&1\\ 1&1\end{bmatrix}\begin{bmatrix}1&0\\ 0&-2\end{bmatrix}\begin{bmatrix}1&-1\\ -1&2\end{bmatrix}$$

The answers are 
$$\lambda_0 = 1, v_0 = \begin{bmatrix}2\\1\end{bmatrix}$$
$$\lambda_1 = -2, v_1 = \begin{bmatrix}1\\1\end{bmatrix}$$

\subsubsection*{4}

\[
\begin{split}
\|tx + (1 - t)y \| ^ 2 &= \langle tx + (1 - t)y, tx + (1 - t)y \rangle \\
                        &= \langle tx, tx\rangle + \langle (1 - t)y, tx\rangle + \langle tx, (1 - t)y\rangle + \langle (1 - t)y, (1 - t)y\rangle    \\
                        &= t^2 \| x\|^2 + (1 - t)^2\| y\| ^ 2 + (t -  t^2)(\langle  x , y \rangle  + \langle y, x \rangle) \\
                        &= t^2(\|x\|^2 + \|y\|^2 - \langle  x , y \rangle - \langle  y , x \rangle) + t(\langle  x , y \rangle + \langle  y , x \rangle-2 \|y\|^2) + \|y\|^2
\end{split}
\]

% We can get 
% \[
% \begin{split}
% &a =  \|x\|^2 + \|y\|^2 - \langle  x , y \rangle - \langle  y , x \rangle \\
% &b = \langle  x , y \rangle + \langle  y , x \rangle-2 \|y\|^2 \\
% &c = \|y\|^2

% \end{split}
% \]

\end{document}
