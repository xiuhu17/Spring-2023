\documentclass{article}
\usepackage[utf8]{inputenc}
\documentclass{article}
\usepackage{indentfirst}
\usepackage{geometry}
\usepackage{ntheorem}
\usepackage{amsmath}
\usepackage{amssymb}
\usepackage{mathtools}
\geometry{left=3cm,right=2.5cm,top=2.5cm,bottom=2.5cm}
% \indent
\newtheorem*{proposition}{Proposition}
\newtheorem*{definition}{Definition}
\newtheorem*{corrolary}{Corrolary}
\newtheorem*{consider}{Consider}
\newtheorem*{theorem}{Theorem}
\newtheorem*{suppose}{Suppose}
\newtheorem*{notice}{Notice}
\newtheorem*{define}{Define}
\newtheorem*{denote}{Denote}
\newtheorem*{lemma}{Lemma}
\newtheorem*{claim}{Claim}
\newtheorem*{proof}{Proof}
\newtheorem*{case}{Case}
\newtheorem*{skill}{Skill}
\newtheorem*{axiom}{Axiom}
\newtheorem*{algorithm}{Algorithm}
\def\theo{\begin{theorem}}
\def\ax{\begin{axiom}}
\def\alg{\begin{algorithm}}
\def\pro{\begin{proof} }
\def\cla{\begin{claim}}
\def\sk{\begin{skill}}
\def\prop{\begin{proposition}}
\def\defi{\begin{definition}}
\def\lem{\begin{lemma}}
\def\cor{\begin{corrolary}}
\def\den{\begin{denote}}
\def\define{\begin{define}}
\def\supp{\begin{suppose}}
\def\enu{\begin{enumerate} \end{enumerate}}
\def\RR{\mathbb{R}}
\def\ZZ{\mathbb{Z}}
\def\NN{\mathbb{N}}
\def\QQ{\mathbb{Q}}
\def\CC{\mathbb{C}}
\def\FF{\mathbb{F}}
\def\KK{\mathbb{K}}
\def\calBB{\mathcal{B}}
\def\calCC{\mathcal{C}}
\def\calLL{\mathcal{L}}
\def\calPP{\mathcal{P}}
\def\implies{\Longrightarrow}
\def\bfit#1{\textit{\textbf{#1}}}
\DeclareSymbolFont{largesymbolsA}{U}{txexa}{m}{n}
\DeclareMathSymbol{\varprod}{\mathop}{largesymbolsA}{16}
\title{Math 484}
\author{Zhihao Wang}
\date{January 2023}

\begin{document}

\maketitle

\subsection*{1}
\subsubsection*{(a)}
\[
\begin{split}
\begin{bmatrix} -1 & 0 & 0 \\ 0 & 2 & 0 \\ 0 & 0 & 1\end{bmatrix}
\end{split}
\]

Since this matrix is diagonal matrix, so the eigenvalues are on its diagonal. Since $-1 < 0$, $2, 1 > 0$. According to theorem 3.2, it is indefinite. 
\subsubsection*{(b)}
\[
\begin{split}
\begin{bmatrix} 2 & 1 & 4 \\ 1 & 3 & 1 \\ 4 & 1 & -1\end{bmatrix}
\end{split}
\]

Since this matrix is symmetric, so we can apply theorem 3.2 to it, firs we need to get its eigenvalues.
\[
\begin{split}
& \det(\begin{bmatrix} 2 & 1 & 4 \\ 1 & 3 & 1 \\ 4 & 1 & -1\end{bmatrix} - \lambda \begin{bmatrix} 1 & 0 & 0 \\ 0 & 1 & 0 \\ 0 & 0 & 1\end{bmatrix}) = -\lambda^3+4\lambda^2+17\lambda-47 = 0 \\ & \implies \lambda_1 \approx 2.2 , \lambda_2 \approx -3.8, \lambda_3 \approx 5.5
\end{split}
\]

Since $\lambda_2 < 0$, $\lambda_1 > 0 , \lambda_3 > 0$. According to theorem 3.2, it is indefinite. 
\subsubsection*{(c)}
\[
\begin{split}
\begin{bmatrix} 1 & -2 & 0 \\ -2 & 4 & 0 \\ 0 & 0 & 3\end{bmatrix}
\end{split}
\]
Since this matrix is symmetric, so we can apply theorem 3.2 to it, firs we need to get its eigenvalues.
\[
\begin{split}
& \det(\begin{bmatrix} 1 & -2 & 0 \\ -2 & 4 & 0 \\ 0 & 0 & 3\end{bmatrix} - \lambda \begin{bmatrix} 1 & 0 & 0 \\ 0 & 1 & 0 \\ 0 & 0 & 1\end{bmatrix}) = -\lambda^3+8\lambda^2-15\lambda \\ & \implies \lambda_1 \approx 0 , \lambda_2 \approx 3, \lambda_3 = 5 
\end{split}
\]
Since $\lambda_2 = 0$, $\lambda_1 > 0 , \lambda_3 > 0$. According to theorem 3.2, it is positive semi-definite.
\subsection*{2}
My first intuition is,
\[
\begin{split}
\begin{bmatrix} 1 & 2 & -1 \\ 2 & 4 & -2 \\ -1 & -2 & 1\end{bmatrix} 
\end{split}
\]
\textbf{Explanation:} According to the equation from slide,

$$X^TAX = \sum \limits_{i} \sum \limits_{j} A_{ij}x_ix_j$$

We can put corresponding coefficient to the position of matrix, that is,
\[
\begin{split}
(x_1 + 2x_2 - x_3)^2 = \ &1x_1x_1+2x_1x_2-1x_1x_3+ \\ 
                            &2x_2x_1 +4x_2x_2-2x_2x_3+ \\
                            &-1x_3x_1-2x_3x_2+x_3x_3
\end{split}
\]

Since $\forall X \in \mathcal{R}^3$, $X^TAX \ge 0$, it must be  positive semi-definite.


\subsection*{3}
$$x^2 + y^3 - 3xy$$

We can get the first derivative as vector, and set it to $0$

$$\begin{bmatrix} 2x - 3y \\ 3y^2 - 3x\end{bmatrix} = 0 \implies (x, y) \in \{(0, 0), (\frac{9}{4}, \frac{3}{2}) \}$$

Now, we can get the second derivative as Hessian matrix,
$$\begin{bmatrix} 2 & -3 \\ -3 & 6y\end{bmatrix} $$

Plug in to different critical points, get two different matrices, 
$$\begin{bmatrix} 2 & -3 \\ -3 & 0\end{bmatrix} , \begin{bmatrix} 2 & -3 \\ -3 & 9\end{bmatrix} $$

Get the eigenvalues,

$$
\begin{cases*}
\lambda_1 = 1+\sqrt{10}, \lambda_2=1-\sqrt{10}\\
\lambda_1=\frac{11+\sqrt{85}}{2}, \lambda_2=\frac{11-\sqrt{85}}{2}
\end{cases*}
$$

According to theorem 3.2, the first matrix is indefinite, according to theorem 4.2, it is a saddle point, \\
According to theorem 3.2, the second matrix is positive definite, according to theorem 2.1, it is a strict local minimizer.

So we get $(0, 0)$ as saddle point, and $ (\frac{9}{4}, \frac{3}{2})$ is a strict local minimizer.

\subsection*{5 Proof}
\subsubsection*{(a)}

$$\forall X \in \mathcal{R}^n, X^TAX = X^TB^TBX = \langle BX, BX\rangle = \|BX \|^2_2 \ge 0$$

So matrix $A$ is positive semi-definite.

\subsubsection*{(b)}
Since $A$ is positive semi-definite, $A$ is also symmetric, applying spectrum theorem on self-adjoint matrix, we can get,

$$A = U^{*}DU = U^TDU$$

Where $D = \diag{(\lambda_1, ..., \lambda_n)$. Since $A$ is positive semi-definite, all its eigenvalues are non-negative according to theorem 3.2, we can write $D$ as

$$D = \sqrt{D}\sqrt{D} =\sqrt{D} \sqrt{D}^T$$

Where,
$$\sqrt{D}_{ij} = \sqrt{D_{ij}}$$

So we get,

$$A = U^T\sqrt{D} \sqrt{D}^TU = (\sqrt{D}^TU)^T\sqrt{D}^TU = B^TB$$

So,

$$B = \sqrt{D}^TU$$


\end{document}
