\documentclass{article}
\usepackage[utf8]{inputenc}
\documentclass{article}
\usepackage{indentfirst}
\usepackage{geometry}
\usepackage{ntheorem}
\usepackage{amsmath}
\usepackage{amssymb}
\usepackage{amsmath}% http://ctan.org/pkg/amsmath
\newcommand{\notimplies}{%
  \mathrel{{\ooalign{\hidewidth$\not\phantom{=}$\hidewidth\cr$\implies$}}}}
  
\geometry{left=3cm,right=2.5cm,top=2.5cm,bottom=2.5cm}
% \indent
\newtheorem*{proposition}{Proposition}
\newtheorem*{definition}{Definition}
\newtheorem*{corrolary}{Corrolary}
\newtheorem*{consider}{Consider}
\newtheorem*{theorem}{Theorem}
\newtheorem*{suppose}{Suppose}
\newtheorem*{notice}{Notice}
\newtheorem*{define}{Define}
\newtheorem*{denote}{Denote}
\newtheorem*{lemma}{Lemma}
\newtheorem*{claim}{Claim}
\newtheorem*{proof}{Proof}
\newtheorem*{case}{Case}
\newtheorem*{skill}{Skill}
\newtheorem*{axiom}{Axiom}
\newtheorem*{algorithm}{Algorithm}
\def\theo{\begin{theorem}}
\def\ax{\begin{axiom}}
\def\alg{\begin{algorithm}}
\def\pro{\begin{proof} }
\def\cla{\begin{claim}}
\def\sk{\begin{skill}}
\def\prop{\begin{proposition}}
\def\defi{\begin{definition}}
\def\lem{\begin{lemma}}
\def\cor{\begin{corrolary}}
\def\den{\begin{denote}}
\def\define{\begin{define}}
\def\supp{\begin{suppose}}
\def\enu{\begin{enumerate} \end{enumerate}}
\def\RR{\mathbb{R}}
\def\ZZ{\mathbb{Z}}
\def\NN{\mathbb{N}}
\def\QQ{\mathbb{Q}}
\def\CC{\mathbb{C}}
\def\FF{\mathbb{F}}
\def\KK{\mathbb{K}}
\def\calBB{\mathcal{B}}
\def\calCC{\mathcal{C}}
\def\calLL{\mathcal{L}}
\def\calPP{\mathcal{P}}
\def\implies{\Longrightarrow}
\def\bfit#1{\textit{\textbf{#1}}}
\DeclareSymbolFont{largesymbolsA}{U}{txexa}{m}{n}
\DeclareMathSymbol{\varprod}{\mathop}{largesymbolsA}{16}
\title{Math 413}
\author{Zhihao Wang}
\date{Feb 2023}

\begin{document}

\maketitle

\subsection*{1}

For each term of LHS, we see that,

$$\text{Num of Lattice Path From $(0, 0, 0)$ to $(i, j, n )$} = {n + i + j \choose i, j , n}$$

$$\text{Num of Lattice Path From $(i, j, n + 1)$ to $(n, n , 2n + 1)$} = {3n - i - j \choose n-i, n-j , n}$$

Since ${n + i + j \choose i, j , n}$ represents the number of lattice path from $(0,0,0)$ to $(i, j, n )$ with the last $z$ coordinate at $n$ when $x=i,y=j$.

Therefore, the last part, we can only begin at $(i, j, n + 1)$.

Therefore, LHS represents the total number of lattice path from $(0,0,0)$ to $(n, n , 2n + 1)$, which is exactly the RHS.


\subsection*{2}

\textbf{Construction}

We need to construct an order which can be derived from $\lhd$, which is $\lhd_2$ and prove it is total order.

$$(x_1, y_1) \lhd (x_2, y_2) \iff x_1 \le x_2 \land y_1 \le y_2 \implies (x_1 \le x_2 \land y_1 = y_2 ) \lor y_1 < y_2 \iff (x_1, y_1) \lhd_2 (x_2, y_2)$$

\textbf{Proof}

It is reflective, since $(x, y)$, $x \le x$, so  $(x \le x \land y = y )$ is true.

It is symmetric, since if $(x_1, y_1) \lhd_2 (x_2, y_2)$, it is antisymmetric. 

It is transitive, since if $(x_1, y_1) \lhd_2 (x_2, y_2)$, $(x_2, y_2) \lhd (x_3, y_3)$, we cna have $(x_1, y_1) \lhd_2 (x_3, y_3)$

It is total order, so it is linear extention.

\end{document}
