\documentclass{article}
\usepackage[utf8]{inputenc}
\documentclass{article}
\usepackage{indentfirst}
\usepackage{geometry}
\usepackage{ntheorem}
\usepackage{amsmath}
\usepackage{amssymb}
\usepackage{amsmath}% http://ctan.org/pkg/amsmath
\newcommand{\notimplies}{%
  \mathrel{{\ooalign{\hidewidth$\not\phantom{=}$\hidewidth\cr$\implies$}}}}
  
\geometry{left=3cm,right=2.5cm,top=2.5cm,bottom=2.5cm}
% \indent
\newtheorem*{proposition}{Proposition}
\newtheorem*{definition}{Definition}
\newtheorem*{corrolary}{Corrolary}
\newtheorem*{consider}{Consider}
\newtheorem*{theorem}{Theorem}
\newtheorem*{suppose}{Suppose}
\newtheorem*{notice}{Notice}
\newtheorem*{define}{Define}
\newtheorem*{denote}{Denote}
\newtheorem*{lemma}{Lemma}
\newtheorem*{claim}{Claim}
\newtheorem*{proof}{Proof}
\newtheorem*{case}{Case}
\newtheorem*{skill}{Skill}
\newtheorem*{axiom}{Axiom}
\newtheorem*{algorithm}{Algorithm}
\def\theo{\begin{theorem}}
\def\ax{\begin{axiom}}
\def\alg{\begin{algorithm}}
\def\pro{\begin{proof} }
\def\cla{\begin{claim}}
\def\sk{\begin{skill}}
\def\prop{\begin{proposition}}
\def\defi{\begin{definition}}
\def\lem{\begin{lemma}}
\def\cor{\begin{corrolary}}
\def\den{\begin{denote}}
\def\define{\begin{define}}
\def\supp{\begin{suppose}}
\def\enu{\begin{enumerate} \end{enumerate}}
\def\RR{\mathbb{R}}
\def\ZZ{\mathbb{Z}}
\def\NN{\mathbb{N}}
\def\QQ{\mathbb{Q}}
\def\CC{\mathbb{C}}
\def\FF{\mathbb{F}}
\def\KK{\mathbb{K}}
\def\calBB{\mathcal{B}}
\def\calCC{\mathcal{C}}
\def\calLL{\mathcal{L}}
\def\calPP{\mathcal{P}}
\def\implies{\Longrightarrow}
\def\bfit#1{\textit{\textbf{#1}}}
\DeclareSymbolFont{largesymbolsA}{U}{txexa}{m}{n}
\DeclareMathSymbol{\varprod}{\mathop}{largesymbolsA}{16}
\title{Math 413}
\author{Zhihao Wang}
\date{Feb 2023}

\begin{document}

\maketitle

\subsection*{1}

WOLG, we arrange all students in a line from and denote $\{a_1, a_2, ..., a_{10} \}$ as the sum of pens from the first person to the $kth$ one, each person has at least one, so we know
    $$a_1 < a_2 < ... < a_{10}$$

Since the total is $21$,

 $$ 1 \le a_1 < a_2 < ... < a_{10} \le 21$$

Also, we could get,

 $$ 1 + 4 \le a_1 + 4 < a_2 + 4 < ... < a_{10} + 4 \le 21 + 4$$
 
 $$ 1 + 8 \le a_1 + 8 < a_2 + 8 < ... < a_{10} + 8 \le 21 + 8$$

Suppose all $a_k$, $a_k + 4$  and $a_k + 8$ are distinct, we will get 30 distinct value from $1$ to $29$,

By the theorem 3.1.1, we know that there must be two values that are the same, one of the three equations will be hold,

$$a_i = a_j + 4 \  \text{or} \ a_i = a_j + 8 \ \text{or} \ a_i + 4 = a_j + 8$$

Therefore, exists consecutive set of students altogether have exactly 4 or exactly 8 pens


\subsection*{2}

WOLG, we suppose there does not exists a $n + 1$ decreasing sequences. Denote $\{m_1, m_1, ... m_{2n+1}\}$ as the longest decreasing sequence starting from $\{a_1, a_2, ..., a_{2n+1} \}$, 

    $$\frac{2n + 1}{n} > 2$$

By corollary 3.2.2, we know that there must exist $3$ elements of the same value from $\{m_1, m_1, ... m_{2n+1}\}$. Suppose there are $m_{k_1} = m_{k_2} = m_{k_3}$, with  $k_1 < k_2 < k_3$.

Suppose $a_{k_i} > a_{k_{i+1}}$ for $i = 1, i = 2$, we can have a longer decreasing sequences. When we put $a_{k_i}$ at the front of ${a_{k_{i+1}}}$. The length of $m_k$ will be: $m_{k_i} \ge m_{k_{i+1}} + 1$ for $i = 1, i = 2$, which contradict what we have. So,  $a_{k_i} \le a_{k_{i+1}}$. We now have,

$$a_{k_1} \le a_{k_2} \le a_{k_3}$$

We have has an increasing subsequence of length 3.


\subsection*{3}

As we know, the total kinds of subsets of $S$ is,

$$\text{Total} = 2^n$$

WLOG, for each subset $s$, we can find its complement set, $S - s$. The total number of this pair is 

$$\text{Pair} = \frac{2^n}{2} = 2^{n-1}$$

By the theorem 3.1.1, we know that in order to  guarantee  one pair, we must have at least $2^{n-1} + 1$ in total. That is, if we have $2^{n-1} + 1$ in total, there must at least be two subsets in one pair which are completely disjoint.  

\subsection*{4}
It is same if we could prove,

$$r(m, n) > (m-1)(n-1)$$

Suppose $m$ represents by coloring red, $n$ represents by coloring blue. By this equation, we could build a graph with $(n-1)$ graphs of $(m-1)$ vertices which complete red edges inside, and they do not share any vertices. 

We follow the rule that all other edges(except those red edges inside $m-1$ vertices graphs) are blue. That is, all edges connecting two vertices which belong to different  $m-1$ vertices graphs are blue.

We should have at least $n$ vertices which all inside edges are blue. By theorem 3.1.1, there must be two vertices belong to (the same) one of $m-1$ graphs. ($\frac{n}{n-1} > 1$)

By the convention, that edgge must be red. Therefore, if we have $(m-1)(n-1)$ vertices, we can not have $K_{(m-1)(n-1)} \notimplies K_{m}, K_{n}$

Then we have, $r(m, n) > (m-1)(n-1) \implies r(m, n) \ge (m-1)(n-1)+1$

\subsection*{5}

We follow the similar idea that, from $\{1, 2, ..., 11 \}$, join $i$ and $j$ with a red edge if and only if the distance between $i$ and $j$ along the circle is at most $3$. And all other are blue edges.

By this rules, we can have all complete $4$-vertices red graph. Since once we want to have $5$ vertices, the minimum max-distance inside the subgraph is greater or equal than $4$, there must exist at least $1$ blue edge. So we can not have a red subgraph with $5$ vertices. 

Now, consider blue subgraph. As we know, the total distance of the entire graph is $10$, and if we need $3$ vertices, by the corollary 3.2.2, $\frac{10}{3} < 3 + 1$, then there must exits $2$ vertices, which distance less than or equal to $3$. Therefore, that edge, will be red. So we can not have a blue subgraph with $3$ vertices. 

Therefore, $K_{11} \notimplies K_5, K_3$, then we have $r(5, 3) \ge 12$



\end{document}
