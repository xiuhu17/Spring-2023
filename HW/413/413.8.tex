\documentclass{article}
\usepackage[utf8]{inputenc}
\documentclass{article}
\usepackage{indentfirst}
\usepackage{geometry}
\usepackage{ntheorem}
\usepackage{amsmath}
\usepackage{amssymb}
\usepackage{amsmath}% http://ctan.org/pkg/amsmath
\newcommand{\notimplies}{%
  \mathrel{{\ooalign{\hidewidth$\not\phantom{=}$\hidewidth\cr$\implies$}}}}
  
\geometry{left=3cm,right=2.5cm,top=2.5cm,bottom=2.5cm}
% \indent
\newtheorem*{proposition}{Proposition}
\newtheorem*{definition}{Definition}
\newtheorem*{corrolary}{Corrolary}
\newtheorem*{consider}{Consider}
\newtheorem*{theorem}{Theorem}
\newtheorem*{suppose}{Suppose}
\newtheorem*{notice}{Notice}
\newtheorem*{define}{Define}
\newtheorem*{denote}{Denote}
\newtheorem*{lemma}{Lemma}
\newtheorem*{claim}{Claim}
\newtheorem*{proof}{Proof}
\newtheorem*{case}{Case}
\newtheorem*{skill}{Skill}
\newtheorem*{axiom}{Axiom}
\newtheorem*{algorithm}{Algorithm}
\def\theo{\begin{theorem}}
\def\ax{\begin{axiom}}
\def\alg{\begin{algorithm}}
\def\pro{\begin{proof} }
\def\cla{\begin{claim}}
\def\sk{\begin{skill}}
\def\prop{\begin{proposition}}
\def\defi{\begin{definition}}
\def\lem{\begin{lemma}}
\def\cor{\begin{corrolary}}
\def\den{\begin{denote}}
\def\define{\begin{define}}
\def\supp{\begin{suppose}}
\def\enu{\begin{enumerate} \end{enumerate}}
\def\RR{\mathbb{R}}
\def\ZZ{\mathbb{Z}}
\def\NN{\mathbb{N}}
\def\QQ{\mathbb{Q}}
\def\CC{\mathbb{C}}
\def\FF{\mathbb{F}}
\def\KK{\mathbb{K}}
\def\calBB{\mathcal{B}}
\def\calCC{\mathcal{C}}
\def\calLL{\mathcal{L}}
\def\calPP{\mathcal{P}}
\def\implies{\Longrightarrow}
\def\bfit#1{\textit{\textbf{#1}}}
\DeclareSymbolFont{largesymbolsA}{U}{txexa}{m}{n}
\DeclareMathSymbol{\varprod}{\mathop}{largesymbolsA}{16}
\title{Math 413}
\author{Zhihao Wang}
\date{Mar 2023}

\begin{document}

\maketitle

\subsection*{1 Proof}

We denote the set of square is $A_1$, the set of cube is $A_2$, we need to calculate,

$$|\bar A_1 \cap \bar A_2| =  |S|  - |A_1| - |A_2| + |A_1 \cap A_2|$$

We know that, 

$$\sqrt{10000} \le 100, \sqrt[3]{1000} \le 21 \implies |A_1| = 100, |A_2| = 21$$

Further, 

$$\sqrt[6]{1000} \le 3 \implies |A_1 \cap A_2| = 4$$

Then, we get,

$$|\bar A_1 \cap \bar A_2| = 10000 - 100 - 21 + 4 = 9883$$

\subsection*{2 Proof}

Suppose we have, $ 0 \le a \le m, 0 \le b \le m, 0 \le c \le 2m$, 

$$a + b + c = 3m$$

where, $A_1$ indicates number of red is greater or equal than $m + 1$, $A_2 $ indicates number of blud is greater or equal than $m + 1$, $A_3$ indicates number of red is greater or equal than $2m + 1$

\[
\begin{split}
|\bar A_1 \cap \bar A_2 \cap \bar A_3| &= |S| - |A_1| - |A_2| - |A_2| + |A_1 \cap A_2| + |A_2 \cap A_3| + |A_1 \cap A_3| - |A_1 \cap A_1 \cap A_3| \\
\end{split}
\]

As we know,

$$|S| = {3m + 3 - 1 \choose 3m}$$

$$a' = a - m - 1, a' + b + c = 2m - 1 \implies |A_1| = {2m - 1 + 3 - 1 \choose 2m - 1}$$
$$b' = b - m - 1, a + b' + c = 2m - 1 \implies |A_2| = {2m - 1 + 3 - 1 \choose 2m - 1}$$
$$c' = c - 2m - 1, a + b + c' = m - 1 \implies |A_3| = {m - 1 + 3 - 1 \choose m - 1}$$

$$a' = a - m - 1, b' = b - m - 1, a' + b' + c = m - 2 \implies |A_1 \cap A_2| = {m - 2 + 3 - 1 \choose m - 2}$$

Other sets are empty,

\[
\begin{split}
|\bar A_1 \cap \bar A_2 \cap \bar A_3| &= {3m + 3 - 1 \choose 3m} - {2m - 1 + 3 - 1 \choose 2m - 1} - {2m - 1 + 3 - 1 \choose 2m - 1} - {m - 1 + 3 - 1 \choose m - 1} + {m - 2 + 3 - 1 \choose m - 2}  \\
\end{split}
\]


\subsection*{3 Proof}

Make all of them identical, denote $A_i$ as $ith$ box is empty, what we need to calculate is 

\[
\begin{split}
\text{Ans} &= |S| - |\bar A_1 \cap \bar A_2 \cap \bar A_3 \cap \bar A_4 \cap \bar A_5|
\end{split}
\]

Therefore, we have,

$$|A_i| = {r + 4 - 1 \choose 3}$$
$$|A_i \cap A_j| = {r + 3 - 1 \choose 2}$$
$$|A_i \cap A_j \cap A_k| = {r + 2 - 1 \choose 1}$$
$$|A_i \cap A_j \cap A_k \cap A_l | = 1 $$
$$|A_i \cap A_j \cap A_k \cap A_l \cap A_m| = 0$$

Then we have,

\[
\begin{split}
\text{Ans} &= {5 \choose 1} {r + 4 - 1 \choose 3} - {5 \choose 2}{r + 3  - 1 \choose 2} + {5 \choose 3}{r + 2 - 1 \choose 1} - {5 \choose 4}
\end{split}
\]

\subsection*{4 Proof}

We can reduce it to a forbidden board, 

$$X_1 = \{ 3 \}$$
$$X_2 = \{ 1, 5 \}$$
$$X_3 = \{ 3 \}$$
$$X_4 = \{ 3, 4 \}$$
$$X_5 = \{ 1, 5 \}$$

We can now find, $(1, 3) (2, 1) (2, 5) (3, 3) (4, 3) (4, 4) (5, 1) (5, 5)$, by theorem 6.4.1.

We partition into two parts, 

$$F_1 = (1,3), (3,3), (4,3), (4,4)$$
$$F_2 = (2,1), (2,5), (5,1), (5,5)$$

Further, $r_1 = 8, r_2 = 5 + 4 + 3 + 4 +4 = 20, r_3 = 4 * 4 = 16, r_4 = 4, r_5 = 0$

\[ 
\begin{split}
p(X_1, X_2, X_3, X_4, X_5) = 5! - 8 * 4! + 20 * 3! - 16 * 2! + 4 = 20
\end{split}
\]



\subsection*{5 Proof}

We can sort the subsets, make sure that $x_i < x_{i+1}$ for each element, assume $a_i = x_{i+1} - x_i \ge 3$,

$$a_1 + a_2 + ... a_{k-1} = t, t \in [3k-3, n - 1]$$

Then, we can have $b_i = a_i - 3 \ge 0$,

$$b_1 + \cdots + b_{k-1} = t - 3k + 3 = p, p \in [0, n - 3k + 2]$$

For each p, we have, ${p + k - 1 - 1 \choose p}$, so we get,

$$\text{Ans} = \sum \limits_{p=0}^{n-3k+2}{p + k - 1 - 1 \choose p} \left( n - p - 3k + 3\right) = {n - 2k + 2 \choose k}$$

\end{document}
