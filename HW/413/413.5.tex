\documentclass{article}
\usepackage[utf8]{inputenc}
\documentclass{article}
\usepackage{indentfirst}
\usepackage{geometry}
\usepackage{ntheorem}
\usepackage{amsmath}
\usepackage{amssymb}
\usepackage{amsmath}% http://ctan.org/pkg/amsmath
\newcommand{\notimplies}{%
  \mathrel{{\ooalign{\hidewidth$\not\phantom{=}$\hidewidth\cr$\implies$}}}}
  
\geometry{left=3cm,right=2.5cm,top=2.5cm,bottom=2.5cm}
% \indent
\newtheorem*{proposition}{Proposition}
\newtheorem*{definition}{Definition}
\newtheorem*{corrolary}{Corrolary}
\newtheorem*{consider}{Consider}
\newtheorem*{theorem}{Theorem}
\newtheorem*{suppose}{Suppose}
\newtheorem*{notice}{Notice}
\newtheorem*{define}{Define}
\newtheorem*{denote}{Denote}
\newtheorem*{lemma}{Lemma}
\newtheorem*{claim}{Claim}
\newtheorem*{proof}{Proof}
\newtheorem*{case}{Case}
\newtheorem*{skill}{Skill}
\newtheorem*{axiom}{Axiom}
\newtheorem*{algorithm}{Algorithm}
\def\theo{\begin{theorem}}
\def\ax{\begin{axiom}}
\def\alg{\begin{algorithm}}
\def\pro{\begin{proof} }
\def\cla{\begin{claim}}
\def\sk{\begin{skill}}
\def\prop{\begin{proposition}}
\def\defi{\begin{definition}}
\def\lem{\begin{lemma}}
\def\cor{\begin{corrolary}}
\def\den{\begin{denote}}
\def\define{\begin{define}}
\def\supp{\begin{suppose}}
\def\enu{\begin{enumerate} \end{enumerate}}
\def\RR{\mathbb{R}}
\def\ZZ{\mathbb{Z}}
\def\NN{\mathbb{N}}
\def\QQ{\mathbb{Q}}
\def\CC{\mathbb{C}}
\def\FF{\mathbb{F}}
\def\KK{\mathbb{K}}
\def\calBB{\mathcal{B}}
\def\calCC{\mathcal{C}}
\def\calLL{\mathcal{L}}
\def\calPP{\mathcal{P}}
\def\implies{\Longrightarrow}
\def\bfit#1{\textit{\textbf{#1}}}
\DeclareSymbolFont{largesymbolsA}{U}{txexa}{m}{n}
\DeclareMathSymbol{\varprod}{\mathop}{largesymbolsA}{16}
\title{Math 413}
\author{Zhihao Wang}
\date{Feb 2023}

\begin{document}

\maketitle

\subsection*{1}

By using the equation 5.11,

$$n(1+x)^{n-1} = \sum \limits_{k=1}^{n} k {n \choose k} x^{k-1}$$

If we plug in $x = -1, n = 100$, we can have

\[
\begin{split}
1{100 \choose 1} - 2{100 \choose 2} + ... - 100{100 \choose 100} = 100 \times 0^{100 - 1} = 0
\end{split}
\]

\subsection*{2}

By theorem 5.2.2, we have
$$(1+x)^{n} = \sum \limits_{k=0}^{n} {n \choose k} x^{k}$$

Now, if we integrate both side, we have,
\[
\begin{split}
\int (1+x)^{n} dx = \int \sum \limits_{k=0}^{n} {n \choose k} x^{k} dx =  \sum \limits_{k=0}^{n} \int {n \choose k} x^{k} dx
\end{split}
\]

Then we have,
\[
\begin{split}
\frac{1}{n+1} (x+1)^{n+1}  -\frac{1}{n+1} = \sum \limits_{k=0}^{n} \frac{1}{k+1} {n \choose k} x^{k+1} 
\end{split}
\]

Now, we assign $x = -1$, and we have

\[
\begin{split}
1 -\frac{1}{2}{n \choose 1} + \frac{1}{3}{n \choose 2} - ... + (-1)^n\frac{1}{n+1}{n \choose n} = - \frac{1}{n+1}
\end{split}
\]


\subsection*{3}
From binomial theorem, we have,

$$(x+1)^{n} = \sum \limits_{k=0}^{n} {n \choose k} x^k$$

If we take derivative on the both side we have,

$$n(1+x)^{n-1} = \sum \limits_{k=0}^{n} k {n \choose k} x^{k-1}$$

If we multiply both side with $x$, we have,

$$nx(1+x)^{n - 1} = \sum \limits_{k=0}^{n} k {n \choose k} x^{k}$$

then, we take derivative again, we have,

$$n\left[ (x+1)^{n-1} + (n-1)x(1+x)^{n-2}\right] = \sum \limits_{k=0}^{n} k {n \choose k} x^{k - 1}$$

Assign $x = 1$, we have

$$n\left[ 2^{n-1} + (n-1)2^{n-2} \right] = \sum \limits_{k=0}^{n} k^2 {n \choose k} $$

\textbf{Combinatorics  Proof}

\[
\begin{split}
\sum \limits_{k = 0}^{n}k^2{n \choose k} &= \sum \limits_{k = 0}^{n} nk{n - 1 \choose k - 1} \\
                                        &= n \left [\sum \limits_{k = 0}^{n} k{n - 1 \choose k - 1} \right ]\\
                                        &= n \left [\sum \limits_{k = 0}^{n} (k - 1){n - 1 \choose k - 1} +  {n - 1 \choose k - 1}\right ] \\
                                        &= n \left [\sum \limits_{k = 0}^{n} (k - 1){n - 1 \choose k - 1}\right ]  +  n \left [\sum \limits_{k = 0}^{n} {n - 1 \choose k - 1}\right ] \\
                                         &= n(n-1) \left [\sum \limits_{k = 0}^{n}{n - 2 \choose k - 2}\right ]  +  n \left [\sum \limits_{k = 0}^{n} {n - 1 \choose k - 1}\right ] \\
                                         &= n(n-1)2^{n-2} + n2^{n-1}
\end{split} 
\]


\end{document}
