\documentclass{article}
\usepackage[utf8]{inputenc}
\documentclass{article}
\usepackage{indentfirst}
\usepackage{geometry}
\usepackage{ntheorem}
\usepackage{amsmath}
\usepackage{amssymb}

\geometry{left=3cm,right=2.5cm,top=2.5cm,bottom=2.5cm}
% \indent
\newtheorem*{proposition}{Proposition}
\newtheorem*{definition}{Definition}
\newtheorem*{corrolary}{Corrolary}
\newtheorem*{consider}{Consider}
\newtheorem*{theorem}{Theorem}
\newtheorem*{suppose}{Suppose}
\newtheorem*{notice}{Notice}
\newtheorem*{define}{Define}
\newtheorem*{denote}{Denote}
\newtheorem*{lemma}{Lemma}
\newtheorem*{claim}{Claim}
\newtheorem*{proof}{Proof}
\newtheorem*{case}{Case}
\newtheorem*{skill}{Skill}
\newtheorem*{axiom}{Axiom}
\newtheorem*{algorithm}{Algorithm}
\def\theo{\begin{theorem}}
\def\ax{\begin{axiom}}
\def\alg{\begin{algorithm}}
\def\pro{\begin{proof} }
\def\cla{\begin{claim}}
\def\sk{\begin{skill}}
\def\prop{\begin{proposition}}
\def\defi{\begin{definition}}
\def\lem{\begin{lemma}}
\def\cor{\begin{corrolary}}
\def\den{\begin{denote}}
\def\define{\begin{define}}
\def\supp{\begin{suppose}}
\def\enu{\begin{enumerate} \end{enumerate}}
\def\RR{\mathbb{R}}
\def\ZZ{\mathbb{Z}}
\def\NN{\mathbb{N}}
\def\QQ{\mathbb{Q}}
\def\CC{\mathbb{C}}
\def\FF{\mathbb{F}}
\def\KK{\mathbb{K}}
\def\calBB{\mathcal{B}}
\def\calCC{\mathcal{C}}
\def\calLL{\mathcal{L}}
\def\calPP{\mathcal{P}}
\def\implies{\Longrightarrow}
\def\bfit#1{\textit{\textbf{#1}}}
\DeclareSymbolFont{largesymbolsA}{U}{txexa}{m}{n}
\DeclareMathSymbol{\varprod}{\mathop}{largesymbolsA}{16}
\title{Math 413}
\author{Zhihao Wang}
\date{Feb 2023}

\begin{document}

\maketitle

\subsection*{1}
Suppose the number of bagels for each type is $x_1, ..., x_6$, then we can get,

$$x_1 + x_2 + x_3 + x_4 + x_5 + x_6 = 20$$

Since each type of bagels should have equal or more than one, we can have $y_k = x_k - 1$, where $y_k \ge 0$

$$y_1 + y_2 + y_3 + y_4 + y_5 + y_6 = 20 - 6 = 14$$

According to the theorem 2.5.1, we can count the number of combination is,

$${{14 + 6 - 1} \choose {6 - 1}} = 11628$$

The count is 11628.

\subsection*{2}
The beginning can be $\{1, 2, 3, 4, 5, 6, 7 ,8 ,9 ,10 \}$

If we only care about combination, the total number of combination will be, 
$$\text{Total} = {52 \choose 5} = 2598960$$

We need to choose $2$ from $4$ colors, and assign them to those cards,

$${4 \choose 2}$$

Also, the valid count will be,

$$({5 \choose 1} + {5 \choose 2}) \cdot 2 = 30$$

$$\text{count} =  |\{1, 2, 3, 4, 5, 6, 7 ,8 ,9 ,10 \}| \cdot {4 \choose 2} \cdot 30 = 3600$$

The answer is still $\frac{3600}{2598960} = 0.1385 \%$

\subsection*{3}
As we know the total number of subset is,

$$\text{Total} = 2^n$$

Since we want exactly two even integers, number of these different choices is,

$${{n / 2} \choose 2}$$

Now, we need to consider about the odd integers, we can have 
$$2^{n/2}$$

Therefore, the valid count will be,

$${{n / 2} \choose 2} \cdot 2^{n/2}$$

So, we can get the probability as,

$$\frac {{{n / 2} \choose 2} \cdot 2^{n/2}}{2^n} = \frac {{{n / 2} \choose 2}}{2^{2 / n }}$$

\subsection*{4}
The total number of elements is $n$, which means there are $2$ elements chooses from this $n - 2$ elements, we can now partition this into two situations.

First, these two elements are different, then the count of combination will be,

$${n \choose {n - 2}} {{n - 2} \choose 2}$$

Now, we need to count the different permutation, according to theorem 2.4.2,

$$\frac{n!}{2! \cdot 2!}$$

The total number for first case will be,

$${n \choose {n - 2}} {{n - 2} \choose 2}\frac{n!}{2! \cdot 2!}$$

Second, these two elements are the same, then the count of combination will be,

$${n \choose {n - 2}} {{n - 2} \choose 1}$$

Now, we need to count the different permutation, according to theorem 2.4.2,
$$\frac{n!}{3!}$$

The total number for second case will be,

$${n \choose {n - 2}} {{n - 2} \choose 1}\frac{n!}{3!}$$

The total value will be,


$${n \choose {n - 2}} {{n - 2} \choose 2}\frac{n!}{2! \cdot 2!} + {n \choose {n - 2}} {{n - 2} \choose 1}\frac{n!}{3!}$$

\subsection*{5 Proof}

Suppose the total number of students in the class is $n$, and we mark the number of known students of each students as $a_k$ where k means $kth$ student.

The total known students number could be is $\{0, 1, ..., n-1 \}$

Since $0$ and $n - 1$ can not be in the set at the same time, since the relationship is reciprocal.

Therefore, the total known students number could be is either $\{1, ..., n-1 \}$ or $\{0, 1, ..., n-2 \}$.

Total count of students is $n$, and the possible different relation is $n - 1$, By the theorem 3.1.1, there must be two students with same value.

\end{document}
