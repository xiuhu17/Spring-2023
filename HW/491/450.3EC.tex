\documentclass{article}
\usepackage[utf8]{inputenc}
\documentclass{article}
\usepackage{indentfirst}
\usepackage{geometry}
\usepackage{ntheorem}
\usepackage{amsmath}
\usepackage{amssymb}
\usepackage{amsmath}% http://ctan.org/pkg/amsmath
\newcommand{\notimplies}{%
  \mathrel{{\ooalign{\hidewidth$\not\phantom{=}$\hidewidth\cr$\implies$}}}}
  
\geometry{left=3cm,right=2.5cm,top=2.5cm,bottom=2.5cm}
% \indent
\newtheorem*{proposition}{Proposition}
\newtheorem*{definition}{Definition}
\newtheorem*{corrolary}{Corrolary}
\newtheorem*{consider}{Consider}
\newtheorem*{theorem}{Theorem}
\newtheorem*{suppose}{Suppose}
\newtheorem*{notice}{Notice}
\newtheorem*{define}{Define}
\newtheorem*{denote}{Denote}
\newtheorem*{lemma}{Lemma}
\newtheorem*{claim}{Claim}
\newtheorem*{proof}{Proof}
\newtheorem*{case}{Case}
\newtheorem*{skill}{Skill}
\newtheorem*{axiom}{Axiom}
\newtheorem*{algorithm}{Algorithm}
\def\theo{\begin{theorem}}
\def\ax{\begin{axiom}}
\def\alg{\begin{algorithm}}
\def\pro{\begin{proof} }
\def\cla{\begin{claim}}
\def\sk{\begin{skill}}
\def\prop{\begin{proposition}}
\def\defi{\begin{definition}}
\def\lem{\begin{lemma}}
\def\cor{\begin{corrolary}}
\def\den{\begin{denote}}
\def\define{\begin{define}}
\def\supp{\begin{suppose}}
\def\enu{\begin{enumerate} \end{enumerate}}
\def\RR{\mathbb{R}}
\def\ZZ{\mathbb{Z}}
\def\NN{\mathbb{N}}
\def\QQ{\mathbb{Q}}
\def\CC{\mathbb{C}}
\def\FF{\mathbb{F}}
\def\KK{\mathbb{K}}
\def\calBB{\mathcal{B}}
\def\calCC{\mathcal{C}}
\def\calLL{\mathcal{L}}
\def\calPP{\mathcal{P}}
\def\implies{\Longrightarrow}
\def\bfit#1{\textit{\textbf{#1}}}
\DeclareSymbolFont{largesymbolsA}{U}{txexa}{m}{n}
\DeclareMathSymbol{\varprod}{\mathop}{largesymbolsA}{16}
\title{ECE 491 EC}
\author{Zhihao Wang}
\date{Marh 2023}

\begin{document}

\maketitle

\subsection*{1 Proof}

$$Ax_{\in} + Ax_{\bot} = A(I - Q_kQ^T_k + Q_kQ^T_k)x = AIx = Ax$$

\subsection*{2 Proof}

$$T_k Q_k^Tx - \lambda Q_k^T x = (Q_k^TAQ_kQ_k^T - \lambda Q_k^T)x $$

Since $Q_k$ has orthogonal cols, so, $Q_kQ_k^T$ is a $n \times n$ matrix with up-left corner being identity matrix of size $k \times k$, therefore, we have,

$$AQ_kQ_k^Tx = A x_{\in}$$ 

Furhter, 

\[
\begin{split}
(Q_k^TAQ_kQ_k^T- \lambda Q_k^T)x &= Q_k^TAx_{\in}- Q_k^T \lambda x \\
&= Q_k^TAx_{\in}- Q_k^T (Ax_{\in} + Ax_{\bot})\\
&= -Q_k^TAx_{\bot}
\end{split}
\]

\subsection*{3 Proof}

$$\|r \|_2 = \| Hz - \mu z\|_2 = \|U(\Lambda - \mu I)U^T z\|_2$$

By spectrum theorem, we know that $U$ is unitary, therefore, 
$\| Ux\|_2 = \| x\|_2$ since it does not change the norm after transforming the vector,

$$\|U(\Lambda - \mu I)U^T z\|_2 = \| (\Lambda - \mu I)U^T z\|$$

Further, since $U^T$ is also a unitary, we have,

$$\frac{\| (\Lambda - \mu I)U^T z\|_2}{\| z\|_2} = \frac{\| (\Lambda - \mu I) z\|_2}{\| z\|_2}$$

Further, suppose each elements in $z  = \sum_i z_i$, each $z_i$ is a vector with only non-zero element $z_i$

$$\|\Lambda z\|_2 = \|\sum_i \lambda_i z_i\|_2$$

We have,

$$\frac{\| (\Lambda - \mu I) z\|_2}{\| z\|_2} = \frac{\| \sum_i (\lambda_i - \mu ) z_i\|_2}{\| \sum_i 1 \times z_i\|_2} \ge \min \limits_{\lambda \in eig (H)}  |\lambda - \mu |$$

Since we can always find a $\lambda_k \in eig (H)$ which will make the previous equation $ |\lambda - \mu |$ smallest.

\subsection*{4 Proof}

From part 2, we have

$$T_k Q_k^Tx - \lambda Q_k^T x = -Q_k^TAx_{\bot}$$

Now, we consider $z = Q_k^Tx$, $T_k = H$ then we have,

$$\min \limits_{\mu \in eig (T_k)}  |\mu - \lambda | \le \frac{\| -Q_k^TAx_{\bot}\|_2}{\|Q_k^Tx \|_2}$$

Since $Q_k$ is orthogonal, we have

$$\| x_{\in}\|_2 = \| Q_kQ^T_kx\|_2 = \| Q^T_kx\|_2  $$

So, we can get,

$$|\mu - \lambda | \le  \frac{\| -Q_k^TAx_{\bot}\|_2}{\| x_{\in}\|_2} \le \frac{\| -Q_k^T\|_2 \|A\|_2 \|x_{\bot}\|_2}{\| x_{\in}\|_2}  = \|A\|_2 \tan \theta$$


\end{document}
