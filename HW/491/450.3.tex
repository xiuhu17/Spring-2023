\documentclass{article}
\usepackage[utf8]{inputenc}
\documentclass{article}
\usepackage{indentfirst}
\usepackage{geometry}
\usepackage{ntheorem}
\usepackage{amsmath}
\usepackage{amssymb}
\usepackage{amsmath}% http://ctan.org/pkg/amsmath
\newcommand{\notimplies}{%
  \mathrel{{\ooalign{\hidewidth$\not\phantom{=}$\hidewidth\cr$\implies$}}}}
  
\geometry{left=3cm,right=2.5cm,top=2.5cm,bottom=2.5cm}
% \indent
\newtheorem*{proposition}{Proposition}
\newtheorem*{definition}{Definition}
\newtheorem*{corrolary}{Corrolary}
\newtheorem*{consider}{Consider}
\newtheorem*{theorem}{Theorem}
\newtheorem*{suppose}{Suppose}
\newtheorem*{notice}{Notice}
\newtheorem*{define}{Define}
\newtheorem*{denote}{Denote}
\newtheorem*{lemma}{Lemma}
\newtheorem*{claim}{Claim}
\newtheorem*{proof}{Proof}
\newtheorem*{case}{Case}
\newtheorem*{skill}{Skill}
\newtheorem*{axiom}{Axiom}
\newtheorem*{algorithm}{Algorithm}
\def\theo{\begin{theorem}}
\def\ax{\begin{axiom}}
\def\alg{\begin{algorithm}}
\def\pro{\begin{proof} }
\def\cla{\begin{claim}}
\def\sk{\begin{skill}}
\def\prop{\begin{proposition}}
\def\defi{\begin{definition}}
\def\lem{\begin{lemma}}
\def\cor{\begin{corrolary}}
\def\den{\begin{denote}}
\def\define{\begin{define}}
\def\supp{\begin{suppose}}
\def\enu{\begin{enumerate} \end{enumerate}}
\def\RR{\mathbb{R}}
\def\ZZ{\mathbb{Z}}
\def\NN{\mathbb{N}}
\def\QQ{\mathbb{Q}}
\def\CC{\mathbb{C}}
\def\FF{\mathbb{F}}
\def\KK{\mathbb{K}}
\def\calBB{\mathcal{B}}
\def\calCC{\mathcal{C}}
\def\calLL{\mathcal{L}}
\def\calPP{\mathcal{P}}
\def\implies{\Longrightarrow}
\def\bfit#1{\textit{\textbf{#1}}}
\DeclareSymbolFont{largesymbolsA}{U}{txexa}{m}{n}
\DeclareMathSymbol{\varprod}{\mathop}{largesymbolsA}{16}
\title{ECE 491 EC}
\author{Zhihao Wang}
\date{Feb 2023}

\begin{document}

\maketitle

\subsection*{1 Proof}

Since we have the equation,

$$g(x) = x - \frac{f(x)}{f'(x)} = x - \frac{(x - x^*)^mh(x)}{((x - x^*)^mh(x))'}$$

We can further reduce this function into,

$$g(x) = x - \frac{(x - x^*)h(x)}{mh(x) + (x - x^*)h'(x)}$$

\subsection*{2 Proof}

Since we get $g(x)$ from previous one, we can get $g'(x)$ as,

$$A(x) = (x - x^*)h(x) \implies A(x^*) = 0, A'(x^*) = h(x^*)$$

$$B(x) = mh(x) + (x - x^*)h'(x) \implies B(x^*) = mh(x^*), B'(x^*) = (m + 1)h'(x^*)$$

Since,

$$g(x) = x - \frac{A(x)}{B(x)} \implies g'(x) = 1 - \frac{A'(x)B(x) - B'(x)A(x)}{B^2(x)}$$

So we have,

$$g(x^*) = 1 - \frac{h(x^*)mh(x^*)}{m^2h^2(x^*)} = 1 - \frac{1}{m}$$

Therefore, I have,

$$C = |g'(x^*)| = |1 - \frac{1}{m}| = 1 - \frac{1}{m} $$

\subsection*{3 Proof}

Now we have,

$$g(x) = x - k\frac{A(x)}{B(x)} \implies g'(x) = 1 - k\frac{A'(x)B(x) - B'(x)A(x)}{B^2(x)}$$

Then,

$$g(x^*) = 1 - k\frac{h(x^*)mh(x^*)}{m^2h^2(x^*)} = 1 - \frac{k}{m}$$

Therefore, 

$$C = |g'(x^*)| = |1 - \frac{k}{m}|  = 0 \implies  k = m$$

In order to make it quadratic, make $k = m$. 

\end{document}
