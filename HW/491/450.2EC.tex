\documentclass{article}
\usepackage[utf8]{inputenc}
\documentclass{article}
\usepackage{indentfirst}
\usepackage{geometry}
\usepackage{ntheorem}
\usepackage{amsmath}
\usepackage{amssymb}
\usepackage{amsmath}% http://ctan.org/pkg/amsmath
\newcommand{\notimplies}{%
  \mathrel{{\ooalign{\hidewidth$\not\phantom{=}$\hidewidth\cr$\implies$}}}}
  
\geometry{left=3cm,right=2.5cm,top=2.5cm,bottom=2.5cm}
% \indent
\newtheorem*{proposition}{Proposition}
\newtheorem*{definition}{Definition}
\newtheorem*{corrolary}{Corrolary}
\newtheorem*{consider}{Consider}
\newtheorem*{theorem}{Theorem}
\newtheorem*{suppose}{Suppose}
\newtheorem*{notice}{Notice}
\newtheorem*{define}{Define}
\newtheorem*{denote}{Denote}
\newtheorem*{lemma}{Lemma}
\newtheorem*{claim}{Claim}
\newtheorem*{proof}{Proof}
\newtheorem*{case}{Case}
\newtheorem*{skill}{Skill}
\newtheorem*{axiom}{Axiom}
\newtheorem*{algorithm}{Algorithm}
\def\theo{\begin{theorem}}
\def\ax{\begin{axiom}}
\def\alg{\begin{algorithm}}
\def\pro{\begin{proof} }
\def\cla{\begin{claim}}
\def\sk{\begin{skill}}
\def\prop{\begin{proposition}}
\def\defi{\begin{definition}}
\def\lem{\begin{lemma}}
\def\cor{\begin{corrolary}}
\def\den{\begin{denote}}
\def\define{\begin{define}}
\def\supp{\begin{suppose}}
\def\enu{\begin{enumerate} \end{enumerate}}
\def\RR{\mathbb{R}}
\def\ZZ{\mathbb{Z}}
\def\NN{\mathbb{N}}
\def\QQ{\mathbb{Q}}
\def\CC{\mathbb{C}}
\def\FF{\mathbb{F}}
\def\KK{\mathbb{K}}
\def\calBB{\mathcal{B}}
\def\calCC{\mathcal{C}}
\def\calLL{\mathcal{L}}
\def\calPP{\mathcal{P}}
\def\implies{\Longrightarrow}
\def\bfit#1{\textit{\textbf{#1}}}
\DeclareSymbolFont{largesymbolsA}{U}{txexa}{m}{n}
\DeclareMathSymbol{\varprod}{\mathop}{largesymbolsA}{16}
\title{ECE 491 EC}
\author{Zhihao Wang}
\date{Feb 2023}

\begin{document}

\maketitle

\subsection*{0 Proof}

First we prove it is symmetric,

$$(A^TA)^T = A^TA$$

Second we need to prove it Symmetric positive semi-definite,

$$\forall x \in \mathcal{R}^n, x^TA^TAx =\langle x, A^TAx \rangle = \| Ax\|_2^2 \ge 0$$

Third, we prove that there is no zero eigenvalue in order to prove it is SPD, $\forall x \in \mathcal{R}^n $

$$x^TA^TAx =\langle x, A^TAx \rangle = \| Ax\|_2^2 = 0 \iff \|Ax \|_2 = 0$$

Since $A$ is full (col)rank, each col is linear independent, we know that,

$$\|Ax \|_2 = 0 \iff x = 0$$

Therefore, 
$$x^TA^TAx = 0 \iff x = 0 \implies \text{A is positive definite}$$


\subsection*{1 Proof}

First define the eigenvalue/singular value of SPD matrix as $\sigma_1, ..., \sigma_n, \sigma_i < \sigma_{i - 1}$. Since it is SPD, by spectrum theorem, its eigenvalue are exactly its singular value. 

If we want to make $fl(M)$ to be indefinite, we only need to make sure 

$$\|fl(M) - M \|_2 = \| \delta M \|_2> \sigma_n$$

We can write $M = U \Sigma U^*$, with $u_n$ as its corresponding $\sigma_n$ singular vector(also is eign vector). Denote $v_n = \frac{u_n}{\|u_n\|_2}$. Assign $\delta M = \epsilon_{mach} \|M \|_2 v_n v_n^T$, we only need

$$\epsilon_{mach} \| M\|_2 > \sigma_n \implies k(M) > \frac{1}{\epsilon_{mach}}$$

Since, 
$$M = A^TA \implies k(M) = k(A^TA) = k(A)^2 \implies k(A) = O(\epsilon_{mach}^{-\frac{1}{2}})$$


\subsection*{2 Proof}

We can write $\tilde Q = (A + \delta A)\tilde R^{-1} \text{when} \| \delta A\|_2 \le O(\epsilon_{mach})$. Under this circumstance, we find that $\tilde Q$ is nearly orthogonal. we have,

$$\frac{\|\hat M \tilde M \|_2}{\|\hat M \|} \le O(\epsilon_{mach})$$

And Cholesky is backward stable, $\tilde R^{-1}$ will be accurate. Denote $A' = (A + \delta A)$, given $\frac{\|\delta A \|_2}{A} \le O(k(A)\epsilon_{mach})$. By $k(R) = k(A)$

$$\tilde M = A'^T A' \implies \tilde Q = A'\hat R^{-1} \ \text{is orthogonal} \implies \|\hat Q - \tilde Q \|_2 \le O(k(A)^2\epsilon_{mach})$$

Now we can get, $\|I -\hat Q \hat Q^T \|_F \le \|\hat Q(\hat Q -\tilde Q )^T \|_F + O(\epsilon_{mach}^2) = O(k(A)^2\epsilon_{mach})$


\subsection*{3 Proof}

From the conclusion we get in the previous parts, $ \text{eignvalues of }\hat Q^T \hat Q \le O(k(A)^2\epsilon_{mach})$ 

$$k(\hat Q) = O(\frac{1 + k(A) \sqrt{\epsilon_{mach}}}{1 - k(A) \sqrt{\epsilon_{mach}}})$$

This is bounded by CONSTANT. Therefore, the second part of CholeskyQR will maintain orthogonality in $O(\epsilon_{mach})$
\end{document}
