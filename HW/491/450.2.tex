\documentclass{article}
\usepackage[utf8]{inputenc}
\documentclass{article}
\usepackage{indentfirst}
\usepackage{geometry}
\usepackage{ntheorem}
\usepackage{amsmath}
\usepackage{amssymb}
\usepackage{amsmath}% http://ctan.org/pkg/amsmath
\newcommand{\notimplies}{%
  \mathrel{{\ooalign{\hidewidth$\not\phantom{=}$\hidewidth\cr$\implies$}}}}
  
\geometry{left=3cm,right=2.5cm,top=2.5cm,bottom=2.5cm}
% \indent
\newtheorem*{proposition}{Proposition}
\newtheorem*{definition}{Definition}
\newtheorem*{corrolary}{Corrolary}
\newtheorem*{consider}{Consider}
\newtheorem*{theorem}{Theorem}
\newtheorem*{suppose}{Suppose}
\newtheorem*{notice}{Notice}
\newtheorem*{define}{Define}
\newtheorem*{denote}{Denote}
\newtheorem*{lemma}{Lemma}
\newtheorem*{claim}{Claim}
\newtheorem*{proof}{Proof}
\newtheorem*{case}{Case}
\newtheorem*{skill}{Skill}
\newtheorem*{axiom}{Axiom}
\newtheorem*{algorithm}{Algorithm}
\def\theo{\begin{theorem}}
\def\ax{\begin{axiom}}
\def\alg{\begin{algorithm}}
\def\pro{\begin{proof} }
\def\cla{\begin{claim}}
\def\sk{\begin{skill}}
\def\prop{\begin{proposition}}
\def\defi{\begin{definition}}
\def\lem{\begin{lemma}}
\def\cor{\begin{corrolary}}
\def\den{\begin{denote}}
\def\define{\begin{define}}
\def\supp{\begin{suppose}}
\def\enu{\begin{enumerate} \end{enumerate}}
\def\RR{\mathbb{R}}
\def\ZZ{\mathbb{Z}}
\def\NN{\mathbb{N}}
\def\QQ{\mathbb{Q}}
\def\CC{\mathbb{C}}
\def\FF{\mathbb{F}}
\def\KK{\mathbb{K}}
\def\calBB{\mathcal{B}}
\def\calCC{\mathcal{C}}
\def\calLL{\mathcal{L}}
\def\calPP{\mathcal{P}}
\def\implies{\Longrightarrow}
\def\bfit#1{\textit{\textbf{#1}}}
\DeclareSymbolFont{largesymbolsA}{U}{txexa}{m}{n}
\DeclareMathSymbol{\varprod}{\mathop}{largesymbolsA}{16}
\title{ECE 491}
\author{Zhihao Wang}
\date{Feb 2023}

\begin{document}

\maketitle

\subsection*{1}

We can split this, 

\[
\begin{split}
\begin{cases}
A_{11}x_1 + A_{12}x_2 = b_1 \\
A_{21}x_1 + A_{22}x_2 = b_2 
\end{cases}
\end{split}
\]


Then we can solve $x_1$ by using the first equation,
\[
\begin{split}
 & A_{11}x_1 = b_1 - A_{12}x_2 \\
 \implies & A_{21}x_1 = A_{21}A_{11}^{-1}(b_1 - A_{12}x_2)
\end{split}
\]

Then, according to the second equation,

\[
\begin{split}
 & A_{22}x_2 = b_2 - A_{21}x_1 \\
 \implies & A_{22}x_2 = b_2 - A_{21}A_{11}^{-1}(b_1 - A_{12}x_2) \\
 \implies & A_{22}x_2 - A_{21}A_{11}^{-1}A_{12}x_2 = b_2 -  A_{21}A_{11}^{-1}b_1 \\
 \implies & (A_{22} - A_{21}A_{11}^{-1}A_{12})x_2 = b_2 -  A_{21}A_{11}^{-1}b_1 
\end{split}
\]

\subsection*{2}

With the same process, we can find the representation for $x_1$,

\[
\begin{split}
\begin{cases}
A_{11}x_1 + A_{12}x_2 = b_1 \\
A_{21}x_1 + A_{22}x_2 = b_2 
\end{cases}
\end{split}
\]

Then, according to the second equation,

\[
\begin{split}
 & A_{22}x_2 = b_2 -  A_{21}x_1\\
 \implies & A_{12}x_2 = A_{12}A_{22}^{-1}(b_2 - A_{21}x_1)
\end{split}
\]

Then, according to the first equation,
\[
\begin{split}
 & A_{11}x_1 = b_1 - A_{12}x_2 \\
 \implies & A_{11}x_1  = b_1 - A_{12}A_{22}^{-1}(b_2 - A_{21}x_1) \\
 \implies & A_{11}x_1 - A_{12}A_{22}^{-1}A_{21}x_1 =b_1 - A_{12}A_{22}^{-1}b_2 \\
 \implies & (A_{11} - A_{12}A_{22}^{-1}A_{21})x_1 = b_1 -  A_{12}A_{22}^{-1}b_2 
\end{split}
\]


Therefore, we can find,

$$(A - A_{12}A_{22}^{-1}A_{21})x = b -  A_{12}A_{22}^{-1}b_2 $$

According to the given equation, then we assign,

$$A_{12} = U, A_{22}^{-1}A_{21} = V^T, A_{12}A_{22}^{-1}b_2 = 0$$

Also, we use the euqatioin from the first part,






\end{document}
